\documentclass[10pt,a4paper]{article}
\usepackage[latin1]{inputenc}
\usepackage{amsmath}
\usepackage{amsfonts}
\usepackage{amssymb}
\usepackage{graphicx}
\author{Raul Persa, Lukas Vogel}
\title{Setup Description}
\begin{document}
	\maketitle
	
\section*{Used Technology}

\begin{description}
	\item[DBMS:] \textbf{Postgres}  \\
	We used materialized views for data-aggregation. To refresh them without downtime (by using the \texttt{concurrently} keyword), at least version 9.4 is required. \\
	An alternative version of vote-aggregation uses a trigger-based approach. It depends on the new upsert-feature of the postgres 9.5 beta. It is meant as a proof of concept and therefore not enabled by default.
	
	\item[Backend:] \textbf{Django} \\
	Django 1.8.6 with Python 3.5 was used during development. Earlier versions should work aswell.
	Our project also depends on the sslserver and bootstrap3 django packages.

	\item[Database-Adapter:] \textbf{psycopg2} \\
	We decided against using the Django Object-relational mapper, as it wouldn't give us enough control over our queries. We used the psycopg2 module to talk to the database instead.
	
	\item[Frontend:] \textbf{jQuery and Bootstrap}
	
	\item[Load testing:] \textbf{locust} \\
	Locust is a Python-module only running on Python 2.
\end{description}
	
\section*{Loading Data Into The Database}

\begin{description}
	\item[Setting up the schema and static data] \hfill \\
		The database schema is defined in the file
		\texttt{schema.sql}. It can be directly imported into postgres. 
		An easier way to setup the schema is by running the Python3 script \texttt{setup.py}. It automatically imports the schema and handles all of the following tasks:
	
	\begin{description}
		\item[Setting up the vote-insertion functions] \hfill \\
		The data is imported via \texttt{plpgsql}-functions. They are defined in the toplevel-file \texttt{voteinsertion.sql}.
	
		\item[Importing views for election algorithm] \hfill \\
		The views and \texttt{plpgsql}-functions needed for the algorithm are defined in the file \texttt{election-algorithm.sql}.
		
		\item[Importing views for analysis of Wahlkreise] \hfill \\
		Some assorted views to facilitate queries for the Wahlkreis-Overview web page are defined in the file \texttt{wahlkreis-analysis.sql}. 
		
		\item[Importing static data]
		Data like parties, Bundesl�nder, Wahlkreise, Candidates, etc. is extracted from the files in the \texttt{data/} directory and imported by \texttt{setup.py}.
	\end{description}	
	
	\item[Generate votes] \hfill \\
		The Python3 script \texttt{votegenerator.py} generates the right amount of voters and votes for the elections of 2009 and 2013 and imports them into the database. It takes advance of the \texttt{plpgsql}-functions defined in \texttt{voteinsertion.sql} already imported by \texttt{setup.py}.

	
\end{description}


\section*{Running the Wahlsystem}
The subdirectory \texttt{datenbanken-app} is a django-project. It consists of the two applications \texttt{wahlanalyse} handling the analysis part of the system and \texttt{wahl} handling the actual voting.

It can be started by executing: \\
\texttt{python datenbanken-app/manage.py runserver localhost:8000}

The Wahlinformationssystem should then be reachable at: \\
\texttt{localhost:8000/wahlanalyse/2/} or just: \texttt{localhost:8000}\\
where 2 is the election id. 

\end{document}