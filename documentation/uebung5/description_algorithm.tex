\documentclass[10pt,a4paper]{article}
\usepackage[latin1]{inputenc}
\usepackage{amsmath}
\usepackage{amsfonts}
\usepackage{amssymb}
\usepackage{graphicx}
\author{Raul Persa, Lukas Vogel}
\title{Short Description of the Seating Algorithm}
\begin{document}

\maketitle

To guarantee the best possible performance we decided against an iterative method for seat assignment and used the method illustrated in \\ http://www.wahlrecht.de/bundestag/index.htm.

The algorithm is fully implemented in SQL and does not depend on any application logic. Parts that cannot be expressed by pure SQL-Queries are implemented as plpgsql-functions. This guarantees a clean interface between the database and the application layer.

First the algorithm generates views for a variation of sub-tasks like determining:
\begin{itemize}
		\item The winners of a direct mandate \verb|directmandate_winners|
		\item The parties eligible for seats in the Bundestag \verb|parties_in_bundestag|
		\item A Bundesdivisor, specifying the amount of votes needed to gain a seat in the Bundestag \verb|bundesdivisor|
		\item The amount of seats for each party in the Bundestag \verb|total_num_seats| calculated with the Sainte-Lague method and the Bundesdivisor.
\end{itemize}

A plpgsql-function \verb|find_partydivisor| algorithmically determines a party divisor for every party to distribute the seats back to each bundesland and the correspondent landesliste. 
A binary search is used for performance reasons.

The view \verb|members_of_bundestag| finally specifies all members of the Bundestag. Those are:

\begin{itemize}
	\item All \verb|directmandate_winners|
	\item All candidates (that have not already won via direct mandate) \\
	 specified by \verb|remaining_cand_on_ll|.
	 They are a member if they are on a landesliste with a listenplatz lower than the number of remaining seats for a party in a bundesland.
\end{itemize}


The algorithm needs about 200ms to calculate the whole Bundstag-composition on commodity hardware (Laptop with Intel i5-4210U).



\end{document}