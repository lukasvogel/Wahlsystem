\documentclass[10pt,a4paper]{article}
\usepackage[utf8]{inputenc}
\usepackage{amsmath}
\usepackage{amsfonts}
\usepackage{amssymb}
\usepackage{graphicx}
\author{Raul Persa, Lukas Vogel}
\title{Exercise 2 - Product Requirements Document}
\begin{document}
	\maketitle
	
	
	\section*{Functional Requirements}
	
	\begin{itemize}
		\item Analysis of current election
			\begin{itemize}
				\item Composition of the Bundestag by party, taking into consideration: Direkt-, \"Uberhangs- and Ausgleichsmandate
				\item Direktmandate by party
				\item Each vote has to be stored separately but can be aggregated on Wahlkreis-level for faster analysis.
			\end{itemize}
		\item Comparation to former elections
			\begin{itemize}
				\item Compare results of current elections to former lections, especially those from 2009 and 2013
				\item Votes from former elections are not kept.
			\end{itemize}
		\item Voting
			\begin{itemize}
				\item Accept and store votes from people who are eligible to vote.
			\end{itemize}
	\end{itemize}
	
	\section*{Nonfunctional Requirements}
	
	\begin{description}
		\item[Performance] Voting, evaluation and analysis has to happen in near realtime.
		\item[Scalability] The system must handle the votes of 60 Million Wahlberechtigte on election day. 
		After the election has closed, the system has to present analytics in realtime to all interested citizens.
		\item[Information privacy] The personal information of all voters and candidates has to be secure under all circumstances.
		\item[Robustness] Loss of power, hardware or software crashes must not lead to a loss of votes.
		\item[Safety] The system has to be safe from intrusion. Only Wahlberechtigte are allowed to vote. They may vote exactly once per election.
		\item[Compliance] The system has to be compliant with the Bundeswahlger\"ateverordnung (BWahlGV) .
	\end{description}
	
	\section*{User Interface}
	
	\begin{itemize}
		\item Voting
			\begin{itemize}
				\item Only one vote per person allowed
				\item The first vote and/or the second vote can be marked as invalid indiviudally.
				\item Order of first and second vote not specified
				\item Neutral presentation of all options (i.e. no default values)
			\end{itemize}
		\item Analysis
			\begin{itemize}
				\item An easy to use web application allows the user to view statistics of the election as well as the items specified in the analysis-part of the functional requirements.
			\end{itemize}
	\end{itemize}
	
	\section*{Acceptance Criteria}
	\begin{itemize}
		\item All functional requirements are fulfilled.
		\item Scalability
			\begin{itemize}
				\item An input of 150 million votes can be handled in 12 hours.
				\item After voting has ended:100 million analyses can be shown over the next 6 hours. 
			\end{itemize} 
		\item Performance
			\begin{itemize}
				\item Vote has to be registered in less than 20 seconds
				\item Calculation of the current election status in less than 10 minutes
				\item A webpage, showing the current election status has to be served in less than 20 seconds.
			\end{itemize}
		\item Robustness
			\begin{itemize}
				\item No votes are lost after power loss, resetting the system.
			\end{itemize}
	\end{itemize}
\end{document}